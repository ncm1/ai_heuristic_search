
\section{(e,f,g)Heuristic Results and Comparisons}

\subsection{Experimental Results}
\newcommand{\ucstime}{21.9412 ms}
\newcommand{\ucsnode}{12907}
\newcommand{\ucslength}{179.66}
\newcommand{\ucscost}{106.6067}


\newcommand{\aStarHOnetime}{21.6008 ms}
\newcommand{\aStarHOnenode}{10374}
\newcommand{\aStarHOnelength}{0.000  \%}
\newcommand{\aStarHOnecost}{0.000 \%}

\newcommand{\aStarHTwotime}{2.9807 ms}
\newcommand{\aStarHTwonode}{460}
\newcommand{\aStarHTwolength}{-5.755  \%}
\newcommand{\aStarHTwocost}{23.6357 \%}

\newcommand{\aStarHThreetime}{0.9796 ms}
\newcommand{\aStarHThreenode}{217}
\newcommand{\aStarHThreelength}{-28.877 \%}
\newcommand{\aStarHThreecost}{36.2839 \%}

\newcommand{\aStarHFourtime}{1.8209 ms}
\newcommand{\aStarHFournode}{754}
\newcommand{\aStarHFourlength}{-16.932 \%}
\newcommand{\aStarHFourcost}{18.6720 \%}

\newcommand{\aStarHFivetime}{3.3776 ms}
\newcommand{\aStarHFivenode}{1660}
\newcommand{\aStarHFivelength}{-15.886 \%}
\newcommand{\aStarHFivecost}{9.1820 \%}


%Weighted A Star Table w = 1.25
\newcommand{\aStarWeightedHOnetime}{23.4999 ms}
\newcommand{\aStarWeightedHOnenode}{9486}
\newcommand{\aStarWeightedHOnelength}{-1.747  \%}
\newcommand{\aStarWeightedHOnecost}{0.0662 \%}

\newcommand{\aStarWeightedHTwotime}{4.2639 ms}
\newcommand{\aStarWeightedHTwonode}{417}
\newcommand{\aStarWeightedHTwolength}{-4.386  \%}
\newcommand{\aStarWeightedHTwocost}{25.6196 \%}

\newcommand{\aStarWeightedHThreetime}{1.9081 ms}
\newcommand{\aStarWeightedHThreenode}{136}
\newcommand{\aStarWeightedHThreelength}{ -33.074 \%}
\newcommand{\aStarWeightedHThreecost}{47.9680 \%}

\newcommand{\aStarWeightedHFourtime}{1.4664 ms}
\newcommand{\aStarWeightedHFournode}{288}
\newcommand{\aStarWeightedHFourlength}{-30.669 \%}
\newcommand{\aStarWeightedHFourcost}{55.3635 \%}

\newcommand{\aStarWeightedHFivetime}{0.9395 ms}
\newcommand{\aStarWeightedHFivenode}{376}
\newcommand{\aStarWeightedHFivelength}{-22.821 \%}
\newcommand{\aStarWeightedHFivecost}{27.5237 \%}

%Weighted A Star table w = 2
\newcommand{\aStarWeightedTwoHOnetime}{14.2124 ms}
\newcommand{\aStarWeightedTwoHOnenode}{6773}
\newcommand{\aStarWeightedTwoHOnelength}{-5.143 \%}
\newcommand{\aStarWeightedTwoHOnecost}{2.2443 \%}

\newcommand{\aStarWeightedTwoHTwotime}{2.0119 ms}
\newcommand{\aStarWeightedTwoHTwonode}{398}
\newcommand{\aStarWeightedTwoHTwolength}{ 0.256 \%}
\newcommand{\aStarWeightedTwoHTwocost}{29.7685 \%}

\newcommand{\aStarWeightedTwoHThreetime}{0.4456 ms}
\newcommand{\aStarWeightedTwoHThreenode}{120}
\newcommand{\aStarWeightedTwoHThreelength}{ -34.220\%}
\newcommand{\aStarWeightedTwoHThreecost}{58.8972 \%}

\newcommand{\aStarWeightedTwoHFourtime}{0.5921 ms}
\newcommand{\aStarWeightedTwoHFournode}{131}
\newcommand{\aStarWeightedTwoHFourlength}{ -31.882 \%}
\newcommand{\aStarWeightedTwoHFourcost}{70.2318 \%}

\newcommand{\aStarWeightedTwoHFivetime}{ 0.3917 ms}
\newcommand{\aStarWeightedTwoHFivenode}{127}
\newcommand{\aStarWeightedTwoHFivelength}{-33.819 \%}
\newcommand{\aStarWeightedTwoHFivecost}{51.0543 \%}


\newcommand{\aStarSequentialtime}{10.0905 ms}
\newcommand{\aStarSequentialnode}{817}
\newcommand{\aStarSequentiallength}{-29.967 \%}
\newcommand{\aStarSequentialcost}{38.3617 \%}

\newcommand{\aStarSequentialtimea}{5.4086 ms}
\newcommand{\aStarSequentialnodea}{1637.3400}
\newcommand{\aStarSequentiallengtha}{-24.2569 \%}
\newcommand{\aStarSequentialcosta}{ 32.4471 \%}


\newcommand{\aStarSequentialtimeb}{10.1010 ms}
\newcommand{\aStarSequentialnodeb}{4648.3600}
\newcommand{\aStarSequentiallengthb}{-16.5980 \%}
\newcommand{\aStarSequentialcostb}{20.9319  \%}


\begin{tabu} to 0.98\textwidth { | X[c] | X[c] | X[c] | X[c]| X[c]|}
 \hline
  Algorithm & Mean Time & Mean Nodes & Optimal Path Length & Path Cost \\
 \hline
  Uniform Cost  &  \ucstime & \ucsnode & \ucslength  &\ucscost\\
 \hline
\end{tabu}


\begin{tabu} to 0.98\textwidth { | X[c] | X[c] | X[c] | X[c] | X[c]|}
 \hline
  Algorithm & Mean Time & Mean Nodes & Percent from optimal & Path Cost\\
 \hline
  A* - H1 & \aStarHOnetime & \aStarHOnenode & \aStarHOnelength & \aStarHOnecost \\
 \hline
  A* - H2 & \aStarHTwotime & \aStarHTwonode & \aStarHTwolength & \aStarHTwocost \\
 \hline
  A* - H3 & \aStarHThreetime & \aStarHThreenode & \aStarHThreelength & \aStarHThreecost \\
 \hline
  A* - H4 & \aStarHFourtime & \aStarHFournode & \aStarHFourlength & \aStarHFourcost \\
 \hline
  A* - H5 & \aStarHFivetime & \aStarHFivenode & \aStarHFivelength & \aStarHFivecost \\
 \hline
 \end{tabu}

 \begin{tabu} to 0.98\textwidth { | X[c] | X[c] | X[c] | X[c] | X[c]|}
 \hline
  Algorithm & Mean Time & Mean Nodes & Percent from optimal & Path Cost\\
  \hline
   A* - H1 W=1.25 & \aStarWeightedHOnetime & \aStarWeightedHOnenode & \aStarWeightedHOnelength & \aStarWeightedHOnecost \\
  \hline
   A* - H2 W=1.25 & \aStarWeightedHTwotime & \aStarWeightedHTwonode & \aStarWeightedHTwolength & \aStarWeightedHTwocost \\
  \hline
   A* - H3 W=1.25 & \aStarWeightedHThreetime & \aStarWeightedHThreenode & \aStarWeightedHThreelength & \aStarWeightedHThreecost \\
  \hline
   A* - H4 W=1.25 & \aStarWeightedHFourtime & \aStarWeightedHFournode & \aStarWeightedHFourlength & \aStarWeightedHFourcost \\
  \hline
   A* - H5 W=1.25 & \aStarWeightedHFivetime & \aStarWeightedHFivenode & \aStarWeightedHFivelength & \aStarWeightedHFivecost \\
  \hline
\end{tabu}


\begin{tabu} to 0.98\textwidth { | X[c] | X[c] | X[c] | X[c] | X[c]|}
\hline
 Algorithm & Mean Time & Mean Nodes & Percent from optimal & Path Cost\\
 \hline
A* - H1 W=2  & \aStarWeightedTwoHOnetime & \aStarWeightedTwoHOnenode & \aStarWeightedTwoHOnelength & \aStarWeightedTwoHOnecost \\
\hline
A* - H2 W=2  & \aStarWeightedTwoHTwotime & \aStarWeightedTwoHTwonode & \aStarWeightedTwoHTwolength & \aStarWeightedTwoHTwocost \\
\hline
A* - H3 W=2 & \aStarWeightedTwoHThreetime & \aStarWeightedTwoHThreenode & \aStarWeightedTwoHThreelength & \aStarWeightedTwoHThreecost \\
\hline
A* - H4 W=2  & \aStarWeightedTwoHFourtime & \aStarWeightedTwoHFournode & \aStarWeightedTwoHFourlength & \aStarWeightedTwoHFourcost \\
\hline
A* - H5 W=2  & \aStarWeightedTwoHFivetime & \aStarWeightedTwoHFivenode & \aStarWeightedTwoHFivelength & \aStarWeightedTwoHFivecost \\
\hline
\end{tabu}

\begin{tabu} to 0.98\textwidth { | X[c] | X[c] | X[c] | X[c] | X[c]|}
\hline
 Algorithm: Sequential A* & Mean Time & Mean Nodes & Percent from optimal & Path Cost\\
 \hline
   W1=1.25 W2=2& \aStarSequentialtime & \aStarSequentialnode & \aStarSequentiallength & \aStarSequentialcost \\
   \hline
   
   W1=1.05 W2=1.25& \aStarSequentialtimea & \aStarSequentialnodea & \aStarSequentiallengtha & \aStarSequentialcosta \\
 \hline
 
    W1=1.00 W2=1.1& \aStarSequentialtimeb & \aStarSequentialnodeb & \aStarSequentiallengthb & \aStarSequentialcostb \\
    \hline
\end{tabu}


The memory requirements for each of these heuristics were the same.

\subsection{Comparison}

Mean Time: Average amount of time it took for the search to execute. \\
Mean Nodes: Average amount of nodes that were expanded during the search. \\
Percent from optimal: Average percentage from the optimal path length. \\ 
Path Cost: Average percentage from the optimal path cost. \\

As expected, the consistent and admissible heuristic H1 behaved similarly to the uniform cost search. It expands less nodes than the uniform cost search because the heuristic prioritizes the nodes moving towards the goal than away from the goal. The mean time was also decreased because of the less amount of nodes expanded.
The consistent and admissible heuristic is optimal.

\subsubsection{Comparison of H2-H5}
Comparing the other four heuristics, we found that H3-5 were far from optimal and took the least amount of time. These three heuristics were primarily based on the distance between the node and the goal. H2 also depended on the distance to the goal, but also took into account whether a river was present to determine the heuristic value.

We can see the effect of having a higher weight. Especially since the heuristics are proportional to the node's distance from the goal node, we find a more direct path as the weight increases. This is evident in the weighted A* approach and from a general observation of the heuristics.

H3 was based on expected value, where we took a very rough consideration for the cost of the average path traversal. Each path cost was expected to be around 1.5, when in reality this is not the case because of the presence of rivers. We omitted the probability of the path including a river, because the probability is difficult to estimate and many assumptions would have to be made. We also found it useful to notice the difference between heuristics that take into account the existence of rivers and those that didn't. H2 accounted for rivers where the others did not.

If H2 does not encounter any rivers, it behaves very similarly to H4 and H5. H2 performs better as the weight on the heuristic increases, as H2 is weaker than the other heuristics, so the weight does not affect the search as much.

H3 is expected to perform the worst, because there are only several different maps and 10 different variations within the maps. As the heuristic value was determined based on probability, we expect it's performance to increase with a larger dataset. Through the study of the H3 heuristic, we realize that it is highly probably that the path will be aided greatly by a river and this should be accounted in our heuristic to improve it.

H4 and H5 are very similar, as they are the Manhattan and Euclidean distances. It appears though that path cost for the Euclidean distance is much lower than the path cost for the Manhattan distance. There's a significant difference of about 10 percent while there's not much difference in path length. We can attribute this to how the Euclidean distance expands more nodes, likely because it's heuristic value is also guaranteed to be lower than the Manhattan distance.

On average, H5 has the lowest heuristic value, with H4 and H2 coming close behind, and H3 having the highest heuristic value. This is how the heuristics would rank in terms of minimizing path cost.

\subsubsection{Path Length vs Path Cost}
Generally, we noticed that as the path length is higher, then the path cost is also lower. We recognized a pattern where if a path length is higher, it often means that a path included a river. In this case the path cost often ends up being lower. If the path length is shorter, it often means that the heuristic dominated and caused the path to go almost straight from the start to the goal without many deviations.

\subsubsection{Mean Time}
Overall, we observed that in general the mean time decreases as the mean number of nodes also decreases. This makes sense as less time is spend expanding nodes, the solution can be reached faster.


