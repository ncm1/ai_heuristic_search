\section{Optimization of heuristic algorithm}

\subsection{Overall Optimizations}

When searching for the neighboring nodes to expand to, we can ignore the neighbor that is the parent because the parent was already explored. This saves the computation time of calculating the path and adding/removing from the binary heap.

The table that keeps the explored set of nodes is a 2 dimensional array. Since our grid is a 2d table, each node is assigned a coordinate. In the grid object that contains the graph of the map and all the nodes, we store a two dimensional array of nodes that represents the grid.

Also for keeping the fringe or the frontier of the graph search, we use a priority queue which is equivalent to a priority heap. We desire a minimal time for a search operation, and a binary heap provides this in O(log(n)) time. Finding the highest element also takes O(1), which is also significant in choosing this data structure to store our possible nodes in our frontier.

Since the overall algorithm is the same for all the types of searches, there were no optimizations specific to each type of search. Each search is different in how it calculates the value, which can consist of the actual path cost to the node or include the heuristic as well. IN the abstract algorithm we call an abstract method, where the implementation of this method is the only difference between the searches.