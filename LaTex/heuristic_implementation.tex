\section{Heuristic Proposals}



\subsection{Best admissible/consistent heuristic}

The best heuristic can be achieved by considering the minimum path cost from the start to end goal.

We will consider the shortest path for three cases, and generalize the results 

-A path from the start node to the goal node is a straight line.

The shortest possible path can occur if a river runs straight from the start to the goal node. This is possible because the river can possibly continue straight across the whole grid. This configuration allows the minimal past cost for this situation. The heuristic value of a node, the predicted cost from the node to the goal:
\\ $s$ is the current node, $s_g$ is the goal node coordinates.
\\ $t$ = 0.25: multiplier for the type of block.
\\ $d$ = 1.0: multiplier for direction of movement.

\[h(s) = t * d  * max(|s^y - s_g^y|,|s^x - s_g^x|)\]
\[h(s) = 0.25 * 1 * max(|s^y - s_g^y|,|s^x - s_g^x|)\]


-A path from the start node to goal node has diagonals and straight lines.

We note that it costs less to move non-diagonally on highways than than moving diagonally along non-rivers. The cost of moving diagonally is:
\[h(s) 4 * 0.25 * \sqrt{2} * \sqrt{2} * min(|s^y - s_g^y|,|s^x - s_g^x|))\]

We note that moves can only be made vertically or horizontally on a highway. So we calculate the total amount of vertical and horizontal moves.
\[h(s) = t * d  * (vert_dist + horiz_dist) \]
\[h(s) = 0.25 * 1 * ( |s^y - s_g^y| + |s^x - s_g^x| )\]
We can clearly see that this heuristic is minimized.


\subsection{Inadmissible heuristics}
We argue that inadmissible heuristics should be used for the A* algorithm on this graph because the range of possible values for the path cost in this grid is high. Since the range is high, to select a admissible heuristic, we must shoot low and calculate the heuristic conservatively to not exceed the minimum path cost. Ideally, we wish for the minimum path cost to be close to the average, or the range to be smaller so that the heuristic value is not much smaller than the cost.
Condition:
\[h(s) <= c(s,s') + h(s') \]
\[h(s) - h(s') <= min(c(s,s'))\]
To find the maximal heuristic that fits this condition, we must find the minimum possible cost value.
With a high range of values for the cost we find that:
\[h(s) - h(s') << average(c(s,s'))\]


In other words, the case where the path is strictly on a river or highway is very unlikely, yet it is still possible. This in turn reduces the maximum heuristic value to force the heuristic to be admissible, and in turn renders the heuristic value to minimally affect the estimated path cost. This is why the admissible and consistent A* implementation almost exactly follows the Uniform Cost Search algorithm.

\subsubsection{H2: Type-based Heuristic}
Currently, the abstract algorithm tends to follow rivers because the path cost to these rivers is small. However this is computationally expensive, as these nodes will have to continuously be expanded along and not along the river. We can argue that the path should follow the river as long as this does not increase the Euclidian distance from the current node to the goal.

If the current node is on a river, then the heuristic can be calculated to be:
\[h(s) = 0.38 * 1 * ( |s^y - s_g^y| + |s^x - s_g^x| )\]
If the river takes us away from the goal, then the heuristic will be large enough that a step of 1 will be enough to have a lower path cost than following the river away from the goal.

The result is that nodes along the river going away from the goal will be expanded less, thus reducing computation time when the path is on a river.

When the current node is a hard to traverse cell, we make the assumption that this region will be full, so we avoid these cells by raising the heuristic value of.
We calculate that traversing around a 31x31 grid of hard-to-traverse cells is slightly shorter than traversing through the 31x31 grid. Since there's only a 50 percent chance of a hard-to-traverse cell of being. Note: traveling through the difficult blocks on average will yield a path cost of 1.5.
\\Traversing around the 31x31 block: $ 30 + 30 + \sqrt{2} = 61.41$
\\Traversing through the 31x31 block \[ \sqrt{2} * 30 * (1 + (1 * (0.50))) = 65.76\] 
$61.41 < 65.76$. It is advantageous to travel around the 31x31 grid.
Though it may have a higher path cost to travel around the block, because with a larger number of blocks.
So this heuristic should be designed to have the path best avoid these hard-to-traverse cells, not necessarily avoid them completely.
\[h(s) = 1.5 * 1 * ( |s^y - s_g^y| + |s^x - s_g^x| )\]

When the current node is a regular unblocked cell:

\[h(s) = t1 * d1  * max(vert_dist, horiz_dist) + t2 * d2 * min(vert_dist, horiz_dist)\]
\[h(s) = 1 * 1 *min(|s^y - s_g^y|, |s^x - s_g^x|) + 1 * \sqrt{2} * max(|s^y - s_g^y|, |s^x - s_g^x|)   \]

 $s$ is the current node, $s_g$ is the goal node coordinates. \\
 $t1 = t2 = 1$: multiplier for the type of block. \\
 $d1 = 1.0$ , $d2 = \sqrt{2}$ : multiplier for direction of movement. \\

This heuristic is advantageous because since it uses the current node type, it is computationally inexpensive. 

\subsubsection{H3: Expected Value}
This method will take into account the number of blocked cells and the number of hard to traverse cells. The value that's generated from the heuristic will take the probability of . The heuristic value will be the average cost of each step combined with a value thats proportional to the node's distance to the goal.

Average cost of step = summation of: probability of scenario * cost of scenario

\[average(s) = \sum{P(s)*C(s)}  \]]
\[average(s) = (0.8 * 1 + 8*0.05*0.5*1.5 - 7*0.05*0.5*1.5  )) \]

There are 8 possible moves from a single node. 20 percent of nodes will be blocked, on average leaving 6.4 possible moves from a node. Of

\subsubsection{H4: Neighbor-checking Heuristic}
This heuristic will check it's neighbors to determine it's heuristic value.

\subsubsection{H5: ?}


Calculate the average path cost.